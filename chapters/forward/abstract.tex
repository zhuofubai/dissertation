\newpage
\begin{centering}
  Localizing Faults in Numerical Software Using a Value-Based Causal Model
  LOCALIZING FAULTS IN NUMERICAL SOFTWARE USING A VALUE-BASED CAUSAL MODEL\\
  \vspace{1cm}
  Abstract\\
  by\\
  \vspace{1cm}
  Zhuofu Bai\\
  \vspace{1cm}
\end{centering}



Causal statistical fault localization (CSFL) thechinque has been approved to be effective in localizing software faults with test profiles and outcome. But when applying CSFL on numerical software fault localization, t
 most research on SFL, execution dynamics have been characterized either purely by code-coverage profiles, indicating which statements, branches, or paths were covered by each execution (e.g., [1]), or by indicators of the outcomes of predicates (e.g., [2]) ? both existing branch predicates and predicates inserted specifically to enhance SFL (e.g., ones that compare the values of numeric variables to zero). Although the outcomes of such predicates reflect the values of program variables to some extent, they often do so inadequately for SFL, e.g., because predicates or conditions were omitted mistakenly, because they are hidden in library code, or because few predicates are actually required in the program. Even predicates inserted in a program to enhance SFL are likely to characterize the values of numeric variables inadequately, unless application-specific information is available that indicates which predicates should be inserted. These facts make coverage and predicate-based SFL techniques poorly suited to localizing faults in numerical programs and subprograms having relatively few conditional branches. 

We first investigates the performance of Baah et al?s causal regression model for fault localization when an important precondition for causal inference, called positivity, is violated.  Two kinds of positivity violations are considered: structural and random ones.  We prove that random, but not structural nonpositivity may harm the performance of Baah et al?s causal estimator.  To address the problem of random nonpositivity, we propose a modification to the way suspiciousness scores are assigned.  Empirical results are presented that indicate it improves the performance of Baah et al?s technique. We also present a probabilistic characterization of Baah et al?s estimator, which provides a more efficient way to compute it.

Then we proposed NUMFL. NUMFL is a value-based causal inference model for localizing faults in numerical software.  NUMFL combines causal and statistical analyses to characterize the causal effects of individual numerical expressions on output errors.  Given value-profiles for an expression's variables, NUMFL uses generalized propensity scores (GPSs) or covariate balancing propensity scores (CBPSs) to reduce confounding bias caused by evaluation of other, faulty expressions.  It estimates the average failure-causing effect of an expression using quadratic regression models fit within GPS or CBPS subclasses.  We report on an empirical evaluation of NUMFL involving components from four Java numerical libraries, in which it was compared to five alternative statistical fault localization metrics.  The results indicate that NUMFL is the most effective technique overall. We also found that NUMFL works fairly well with data from failing runs alone.









