\newpage
\begin{centering}
  Localizing Faults in Numerical Software Using a Value-Based Causal Model
  LOCALIZING FAULTS IN NUMERICAL SOFTWARE USING A VALUE-BASED CAUSAL MODEL\\
  \vspace{1cm}
  Abstract\\
  by\\
  \vspace{1cm}
  Zhuofu Bai\\
  \vspace{1cm}
\end{centering}



Causal statistical fault localization (CSFL) technique, for example, Baah et al’s causal regression model, has been approved to be effective in localizing software faults with test profiles and outcome. In most research on CSFL, execution dynamics have been characterized by code-coverage profiles, indicating which statements, branches, or paths were covered by each execution. This coverage based CSFL has two potential problems: (1) The cause effect estimation can be biased when the positivity condition is violated  (2) It poorly suited for localizing faults in numerical programs and subprograms having relatively few conditional branches. 

To solve the above problems, we first investigates the performance of Baah et al’s causal regression model for fault localization when an important precondition for causal inference, called positivity, is violated.  Two kinds of positivity violations are considered: structural and random ones.  We prove that random, but not structural nonpositivity may harm the performance of Baah et al’s causal estimator.  To address the problem of random nonpositivity, we propose a modification to the way suspiciousness scores are assigned.  Empirical results are presented that indicate it improves the performance of Baah et al’s technique. We also present a probabilistic characterization of Baah et al’s estimator, which provides a more efficient way to compute it.

Then we proposed two value-cased causal inference models for localizing faults in numerical software. The first model is NUMFL. NUMFL combines causal and statistical analyses to characterize the causal effects of individual numerical expressions on output errors.  Given value-profiles for an expression's variables, NUMFL uses generalized propensity scores (GPSs) or covariate balancing propensity scores (CBPSs) to reduce confounding bias caused by evaluation of other, faulty expressions.  It estimates the average failure-causing effect (AFCE) of an expression using quadratic regression models fit within GPS or CBPS subclasses.  We report on an empirical evaluation of NUMFL involving components from four Java numerical libraries, in which it was compared to five alternative statistical fault localization metrics.  The results indicate that NUMFL is more effective than baseline techniques. We also found that NUMFL works fairly well with data from failing runs alone. 

The second model is Bayesian Additive Regression Trees (BART) model. Instead of controlling confounding bias with propensity scores, BART model fits a sum of trees structure to approximate the dose response function (DRF) of both treatment variable and the confounding variables. For every unit in the observational data set of an expression, BART model estimate the causal effect of treatment variable on outcome with the fitted sum of trees structure. The average value of the estimated causal effect of each unit in data set is the estimated AFCE of the expression. We compare the performance of BART model with that of NUMFL and five baseline techniques in empirical evaluation. The result shows BART model is the most effect techniques overall.  But the computation cost of BART model is more expensive than NUMFL and other baseline techniques.









