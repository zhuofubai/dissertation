\newpage
\begin{centering}
  Localizing Faults in Numerical Software Using a Value-Based Causal Model
  \\
  \vspace{1cm}
  Abstract\\
  by\\
  \vspace{1cm}
  Zhuofu Bai\\
  \vspace{1cm}
\end{centering}



Causal statistical fault localization (CSFL) techniques,  which applies causal inference techniques to test-execution profiles and test outcomes, estimates the causal effect of individual program elements on the occurrence of failures, and have been found to be effective in localizing software faults. In most research on CSFL, execution dynamics have been characterized by code-coverage profiles that indicate which statements, branches, or paths were covered by an execution. This coverage-based CSFL has two potential problems: (1) causal effect estimation can be biased when an important precondition for causal inference, called positivity, is violated  (2) it is poorly suited for localizing faults in numerical programs and subprograms having relatively few conditional branches.
 
To solve the aforementioned problems, we first investigate the performance of Baah et al's causal regression model for fault localization when the positivity condition $\forall t,\,x\;\Pr (T = t|X = x) > 0$ is violated, where $T$ is a coverage indicator for the target statement and  $X$ is the set of covariates used for confounding adjustment.  Two kinds of positivity violations are considered: structural and random ones.  We prove that random, but not structural nonpositivity may harm the performance of Baah et al's causal estimator.  To address the problem of random nonpositivity, we propose a modification to the way suspiciousness scores are assigned.  Empirical results are presented that indicate it improves the performance of Baah et al's technique. We also present a probabilistic characterization of Baah et al's estimator, which provides a more efficient way to compute it.
 
Then we present two value-based causal inference models for localizing faults in numerical software. The first model is denoted NUMFL. NUMFL combines causal and statistical analyses to characterize the causal effects of individual numerical expressions on output errors.  Given value-profiles for an expression's variables, NUMFL uses generalized propensity scores (GPSs) or covariate balancing propensity scores (CBPSs) to reduce confounding bias caused by confounding variables, which are the evaluation of other, faulty expressions.  It estimates the average failure-causing effect (AFCE) of an expression using quadratic regression models fit within GPS or CBPS subclasses.  We report on an empirical evaluation of NUMFL involving components from four Java numerical libraries, in which it was compared to five alternative statistical fault localization metrics.  The results indicate that NUMFL is more effective than baseline techniques. We also found that NUMFL works fairly well with data from failing runs alone.
 
The second model is based on Bayesian Additive Regression Trees (BART), which are due to Chipman et al. Instead of controlling confounding bias with propensity scores, we use a BART model to approximate the dose-response function (DRF) relating the treatment variable to the output errors, where treatment variable is the result of evaluating a numerical expression. Given a unit in the observational data set, we input it into the fitted BART model, and then increase the value of the treatment variable, but keep the confounding variables unchanged. The causal effect of treatment for that unit is estimated by the change in the output of the BART model. The average value of the estimated causal effect of each unit in the data set is the estimated AFCE of the expression. We compare the performance of BART model with that of NUMFL and five baseline techniques in an empirical evaluation. The result shows BART model is the most effect techniques overall.  




