\newpage
\begin{centering}
  Development of computational approaches for medical image retrieval, disease gene prediction, and drug discovery\\
  \vspace{1cm}
  Abstract\\
  by\\
  \vspace{1cm}
  Yang Chen\\
  \vspace{1cm}
\end{centering}



With the deluge of biomedical data, developing computational approaches for data analysis and interrogation has become a key step in translational biomedical research. It is critical to leverage existing data to ask the right question and design algorithms for specific biomedical applications. In this dissertation, I propose using domain knowledge to guide the data gathering, data fusion and algorithm design in solving specific biomedical problems. I demonstrate the strategy with applications in three distinct contexts.

The first application is retrieving disease manifestation images from the web for supporting patients' self-education and decision making. The challenge is three-fold: heterogeneous irrelevant web images need to be filtered; the positive examples of disease images contain diverse objects and complex backgrounds; and large amounts of manual efforts in generating training data are unaffordable. We observe that detecting disease-affected abnormal organs may greatly reduce the manual labeling efforts. In our approach, we extract the disease-organ semantic relationships from ontologies to guide the organ detection with pre-trained detectors. Comparing with a standard supervised method, we improve the average precision by 4\% while reduce the manual efforts by 85\%.

In the second application, we develop three disease-specific models to detect genetic basis for human diseases. For parasitic infectious diseases, we construct a cross-species genetic network to model host-pathogen interactions and analyze the network to predict disease associated genes. We apply the approach on malaria and demonstrate the potential of the top-ranked genes in guiding anti-malaria drug discovery. For multifactorial diseases, we assume that phenotypic similarity reflects common genetic basis between diseases. We explore a new disease phenotype data source in medical ontologies and construct the Disease Manifestation Network (DMN). Then we integrate multiple phenotype networks with genetic networks to predict genes. We apply the approach on Crohn's disease, and demonstrate the translational potential of the predicted genes in drug discovery. Last, we identify the mutual comorbidity for colorectal cancer and obesity in the comorbidity network to detect genetic basis for the link between the two diseases.

Finally, I present a drug repositioning approach combining disease genetics and phenotypic descriptions for mouse genetic mutations. Disease associated genes have the potential to guide drug discovery. On the other hand, the mouse phenotypes provide knowledge on gene functions, which is impossible to be obtained in human. In our approach, we identify disease-specific mouse phenotypes using well-studied disease genes, and search all FDA-approved drugs for the candidates that share similar mouse phenotype profiles with the disease. We used the approach to predict drugs for Parkinson's disease, and demonstrate significantly improvements comparing with a state-of-art approach based on mouse phenotype data. Overall, I demonstrate the effectiveness of the domain knowledge guided computational approaches in concrete biomedical applications.

In summary, I demonstrate the effectiveness of computational algorithms in translational biomedical research. I demonstrate that my computation-based work have great potential in elucidating disease genetic basis, finding innovative drugs, and improving patient health education.






