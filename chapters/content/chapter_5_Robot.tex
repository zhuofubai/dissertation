\chapter{Studying disease comorbidity network to detect genetic evidences for disease links: application on colorectal cancer and obesity}\label{cancer}

\section{Motivation}

A number of epidemiological studies suggest that
obesity increases the risk of colorectal cancer (CRC)
\cite{calle2003overweight,bardou2013obesity,khaodhiar1999obesity}.
Based on these evidences of co-occurrence,
many genetic factors have been proposed to explain the role of obesity in the development of CRC.
For example, both animal and human studies have demonstrated that the
increased release of insulin and reduced insulin signaling play roles
in obesity and colorectal carcinogenesis \cite{pollak2008insulin,leroith2003insulin,renehan2004insulin}.
Experiments also show that obesity leads to altered level of adipocytokines,
such as Adiponectin \cite{dalamaga2012role,an2012adiponectin,wei2005low}
and leptin \cite{stattin2003plasma,tamakoshi2004leptin}, which may either prevent or foster carcinogenesis.

The mechanism for the association between obesity and CRC is multifactorial and inconclusive
\cite{khaodhiar1999obesity,danese2012role}. Shared comorbidities between
obesity and CRC can provide unique insights into the common genetic basis for the two diseases.
For example, type 2 diabetes is highly correlated with obesity and was identified as a risk factor
for CRC \cite{berster2008type}. A few studies then discovered that genetic factors
of insulin resistance, which occur in type 2 diabetes, contribute in explaining the role of obesity
in CRC \cite{komninou2003insulin}. However, both obesity and CRC are heterogeneous conditions.
Over 40\% of the obese population is not characterized by the presence of insulin
resistance \cite{mesquita2009metabolically}. We hypothesize that systems approaches
to studying the diseases that are phenotypically-significant to
both CRC and obesity may offer new insights into the
common molecular mechanisms between the two interconnected diseases.

Systematic comorbidity studies have been conducted previously,
but mostly focused on pairwise comorbidities and their genetic overlaps.
Rhetsky et al. developed a statistical model to estimate the co-occurrence
relationship for each pair of 160 diseases \cite{rzhetsky2007probing},
and demonstrated that comorbidities are genetically linked. Park et al. \cite{park2009impact}
and Hidalgo et al. \cite{hidalgo2009dynamic}
detected the comorbidities pairs from the Medicare claims
(which only contain senior patients ages 65 or older) with statistical measures.
Roque et al. mined pairwise disease correlations using similar measures from
medical records of a psychiatric hospital \cite{roque2011using}.


In this study, we developed a novel approach to detect diseases
that have strong connections with both obesity and CRC in a comorbidity network.
Specifically, we first mined disease comorbidity relationships from a new data source and constructed
a novel disease comorbidity network. Then we extracted the local network consisting of all the
paths between obesity and CRC, and prioritized the nodes (diseases) that play critical roles
in maintaining the connection between the two diseases (Fig.\ref{crchypothesis}). Substantial literature evidences can support that the top ranked diseases have associations with both obesity and CRC. We investigated the gene expression profiles of a prioritized comorbid disease to facilitate detecting novel genetic basis underlying the link between obesity and CRC. Our approach is generalizable to study the genetic basis for other disease associations.
\begin{figure}[!tpb]
\vspace{-.1cm}
\centerline{\includegraphics[width=0.5\textwidth]{Chap5_crc_hypothesis.eps}}
%\vspace{-5cm}
\caption{Approach to detect the diseases that have strong connections with both obesity and CRC in the comorbidity network. Nodes D1, D2 and D3 were prioritized because they play important roles in maintaining the network structure and the connection. }
\vspace{-0cm}
\label{crchypothesis}
\end{figure}

\section{Data and methods}
Fig.\ref{crcmethod} shows the steps of our approach.
We first mined disease comorbidity relationships from large amounts of patient records
in a public database and constructed a disease comorbidity network.
We then extracted the local comorbidity cluster for obesity and CRC
and prioritize the candidate comorbidity that plays a critical role in connecting the two diseases.
Finally we conducted gene expression meta-analysis to
identify common genes shared by obesity, CRC and the prioritized comorbidity.
\begin{figure}[!tpb]
\vspace{-.1cm}
\centerline{\includegraphics[width=1\textwidth]{Chap5_crc_method.eps}}
%\vspace{-5cm}
\caption{Our approach contains three steps: (1) We constructed a comorbidity network based on data mining; (2) we extracted the local network that contains paths from obesity to CRC, and analyzed the local network to pin point the strong comorbidity for both obesity and CRC; (3) we conducted gene expression meta-analysis to identify common genes shared among obesity, CRC and the comorbidity. }
\vspace{-0cm}
\label{crcmethod}
\end{figure}

\subsection{Construct disease comorbidity network}
\subsubsection{Data sets for comorbidity mining}
The adverse event reports contain records of 3,354,043 patients.
Among all patients, 66\% and 94\% have their age and gender information available.
Figure \ref{distr}(a)-(b) show distributions of age and gender.
Unlike the Medicare system, FAERS contains patients in of ages
from one day to hundreds of years.
The distributions are not severely inclined to particular gender or age levels.
\begin{figure}[!ht]
\vspace{-0.6cm}
\begin{center}
\includegraphics[width=\textwidth]{Chap5_demographics.eps}
\end{center}
\vspace{-0.5cm}
\caption{
{(a) Age distribution of the patients in the adverse event reports. (b) Gender distribution. (c) Distribution of disease semantic types: T047, Disease or Syndrome; T020, Acquired Abnormality; T046, Pathologic Function; T184, Sign or Symptom; T033, Finding; T190, Anatomical Abnormality; T191, Neoplastic Process; T048, Mental or Behavioral Dysfunction; T049, Cell or Molecular Dysfunction; T019, Congenital Abnormality; T037, Injury or Poisoning.}
}
\vspace{-0.5cm}
\label{distr}
\end{figure}

The data represents the diseases that patients have by 10,122 indications of drugs that patients take.
These indication terms include not only diseases, but also treatment procedures, such as surgery; common symptoms, such as pain; and ill-defined events, such as unevaluable events.
We mapped the indication terms to the concept unique identifiers (CUIs)
in Unified Medical Language System (UMLS) and extracted their semantic types.
Figure \ref{distr}(c) listed the distribution of eleven semantic types, in which
the types such as ``disease or syndromes," ``neoplastic process," and ``mental or behavioral dysfunction"
contain disorder concepts.
With the disease data for million of patients, we were able to conduct large-scale comorbidity mining and extract
interesting disease associations.

\subsubsection{Preprocess data}
We developed an automatic pipeline to preprocess the patient-indication pairs (Figure \ref{dataworkflow}).
We mapped all indication terms to CUIs and classified them by semantic types using the UMLS metathesaurus.
Then We selected the identifiers of six semantic types: Mental or Behavioral Dysfunction, Neoplastic Process, Acquired Abnormality, Congenital Abnormality, Disease or Syndrome, and Anatomical Abnormality. We combined the synonyms among terms corresponding to these identifier and removed those only appearing once in the data, since rare diseases may lead to unstable association patterns. Finally, the data contains 3,033,368 links between 2,371,406 patients and 3,994 diseases.
\begin{figure}[!ht]
%\vspace{-0.6cm}
\begin{center}
\includegraphics[width=\textwidth]{Chap5_dataworkflow.eps}
\end{center}
\vspace{-0.5cm}
\caption{
{Automatic pipeline to pre-process the patient-disease data in adverse event reports and mine comorbidity patterns}
}
\vspace{-0.5cm}
\label{dataworkflow}
\end{figure}


\subsubsection{Mine comorbidity patterns}
We explored comorbidity patterns among the 3,994 diseases with association rule mining.
Due to the large number of patients and diseases in the adverse event reports,
exhausting all possible association patterns is computationally impractical.
We applied the frequent pattern growth algorithm,
which uses an tree structure to compress the input and grow
the patterns in a bottom-up manner \cite{han2000mining}.
Previous effort has demonstrated that this algorithm outperforms other popular pattern mining methods,
such as the Apriori algorithm \cite{agrawal1994fast}.
The frequent pattern growth algorithm has also been successfully
applied in biomedical domain to extract drug adverse effects \cite{luo2013mining}.
Association rule mining can flexibly detect strong co-occurrence relationships among sets of diseases, and alleviates the intrinsic bias of traditional comorbidity measures (such as relative risk and $\phi$-correlation) towards rare diseases.

We implemented the algorithm using the Weka java package \cite{hall2009weka}.
The result of the algorithm is a set of patterns indicating how diseases are associated with each other.
The pattern between two sets of diseases is
represented in the form ${\rm{X}} \Rightarrow {\rm{Y}}$,
where $X$ is the pattern body and $Y$ is the pattern head.
For example, $[anxiety, amnesia] \Rightarrow [depression]$
indicates that when patients have anxiety and amnesia,
are also likely to have depression.
Note that though each pattern is directed with an arrow,
they do not indicate causations between diseases,
but represent co-occurrences. To avoid confusion,
we currently ignored the directions of patterns,
 considered all diseases in set $X$ and $Y$ associated.

The mining algorithm requires a few parameters:
the minimum support was set to 0.0008\%, which means
at least 20 patients should have all the diseases
in each pattern at the same time; the maximum
number of diseases in each pattern was set at 3;
and confidence was chosen to measure and rank the patterns.
The confidence score of pattern ${\rm{X}} \Rightarrow {\rm{Y}}$ is defined as:
\begin{equation}
confidence(X \to Y) = |X \cup Y|/|X|\label{arm},
\end{equation}
where $|X \cup Y|$ is the number of patients who have diseases
in both $X$ and $Y$, and $|X|$ is the number of patients who have diseases in $X$.

\subsubsection{Construct disease comorbidity network }
We constructed an undirected and unweighted comorbidity network
based on the result of association rule mining,
which is a list of patterns between two sets of diseases,
represented in the form $x \to y$.
We collected all diseases in the set x and y in each pattern, a
ssuming they have comorbidity relationships with each other,
and established an edge between each pair of diseases in $x \cup y$ to construct the comorbidity network.

\subsection{Prioritize the diseases that have strong associations with both obesity and CRC}
We extracted the local network consisting of  the paths from obesity to CRC in the disease comorbidity network. The local network thus includes the nodes that may represent different aspects of the relationship between obesity and CRC. We implemented breath first search to enumerate the paths, and limited the paths within four steps.
Then we ranked the nodes in the local network, except obesity and CRC, based on how important they are in maintaining the local network structure and the connection between obesity and CRC. We used the degree and betweenness centrality to characterize the importance of each node in the flowing of the network. The degree of a node becomes higher if more paths between obesity and CRC pass through this node. The betweeness evaluates the number of times that the node acts as the bridge along the shortest paths. Removing the nodes with highest degree or betweenness can easily break down the connection between obesity and CRC. We investigated the top ranked diseases based on both ranking methods, and used the unexpected ones to guide the detection of genetic associations between obesity and CRC.

\subsection{Identify gene overlaps through gene expression meta-analysis}
We chose a top ranked disease on the path between obesity and CRC, and then conducted gene expression meta-analysis for the prioritized disease, obesity and CRC, respectively, to detect new genetic explanations for the relationship between obesity and CRC. Gene expression normalized data (SOFT files) were downloaded from NCBI GEO omnibus (GEO, http://www.ncbi.nlm.nih.gov/geo/) using the R package GEOquery \cite{davis2007geoquery}. Then, we performed microarray meta-analyses for each disease independently using the R package MetaDE \cite{wang2012r}. MetaDE implements meta-analysis methods for differential expression analysis, and we used the Fisher's method. Significant differentially expressed genes (DEGs) were selected as those displaying a FDR corrected p-value <0.05. Last, we extracted the common significant genes for the three diseases.

\section{Results}
\subsection{Local disease comorbidity network models the connection between obesity and CRC}
We extracted 7006 comorbidity association rules with the confidence larger than 50\% from the patient records across ten years. The comorbidity network based on these  rules contains 771 nodes and 15,667 edges. Fig.\ref{localnet} shows the local network consisting of all the 119 paths (no longer than four steps) from obesity to CRC. A total of 24 nodes in the local network are the candidate diseases, which have associations with both obesity and CRC, and may indicate different aspects of the relationship between the two diseases.

\subsection{Osteoporosis shows high comorbidity associations with both CRC and obesity}
Table \ref{noderank} shows the top five nodes sorted by degree and betweenness in the local network. In either way of ranking, hypertension, diabetes and hyperlipaemia were in top three and closely related with both obesity and CRC. Substantial literature evidences support that the metabolic syndrome components, hypertension and hyperlipaemia, as well as diabetes have association with obesity and CRC through insulin resistance in substantial literature \cite{khaodhiar1999obesity,pollak2008insulin,leroith2003insulin,renehan2004insulin,komninou2003insulin}. These three disorders also independently increase the risk of CRC and colorectal adenoma \cite{khaodhiar1999obesity,berster2008type,komninou2003insulin}.
The top ranked comorbidities demonstrated the validity of our network analysis approach.
\begin{figure}[!ht]
\vspace{-0.6cm}
\begin{center}
\includegraphics[width=6.5in]{Chap5_crc_localnet.eps}
\end{center}
\vspace{-0.5cm}
\caption{
{The local network that contains all paths from obesity to colorectal cancer in the comorbidity network.}
}
\vspace{-0.5cm}
\label{localnet}
\end{figure}

\begin{table}[h]
\caption{T\lowercase{OP FIVE DISEASE NODES IN THE LOCAL NETWORK THAT CONTAINS ALL PATHS FROM OBESITY TO COLORECTAL CANCER. THE DISEASES WERE RANKED BY DEGREE AND BETWEENNESS, RESPECTIVELY}.}
\label{noderank}
\centering
\begin{tabular}{ccccc}
\hline
\multirow{2}{*}{Rank} & \multicolumn{2}{l}{Ranked by degree} & \multicolumn{2}{l}{Ranked by betweenness} \\
                      & Nodes                  & Degree      & Nodes                  & Betweenness      \\\hline
1                     & Hypertension           & 26          & Hypertension           & 60.2             \\
2                     & Diabetes mellitus      & 24          & Diabetes mellitus      & 55.9             \\
3                     & Hyperlipaemia          & 22          & Hyperlipaemia          & 35.2             \\
4                     & Osteoporosis           & 14          & Osteoporosis           & 12.3             \\
5                     & Hypothyroid            & 14          & Hypothyroid            & 9.5 \\\hline
\end{tabular}
\end{table}

Significantly, osteoporosis was ranked highly by both centrality ranking methods. Epidemiological studies suggested an inverse association between bone mineral density and CRC \cite{nelson2002bone},
colon cancer among postmenopausal women \cite{ganry2008bone},
and colorectal adenoma \cite{nock2011higher}.
On the other hand, patients of obesity and osteoporosis may share common genetic and environmental factors \cite{zhao2007relationship}.
Different from previous studies, our result shows that osteoporosis is crucial for the association between CRC and obesity. Fig.\ref{localnet} shows the paths of obesity-osteoporosis-CRC. We further investigate the gene expression profiles of osteoporosis patients to gain novel insight of the genetic basis for the link between obesity and CRC.
\begin{figure}[!ht]
\vspace{-0.6cm}
\begin{center}
\includegraphics[width=0.7\textwidth]{Chap5_crc_osteo.eps}
\end{center}
\vspace{-0.5cm}
\caption{
{The paths from obesity to colorectal cancer that pass through osteoporosis.}
}
\vspace{-0.5cm}
\label{localnet}
\end{figure}

\subsection{Innovative genes shared among osteoporosis, obesity and CRC are detected using gene expression meta-analysis}

We downloaded five microarray series (GSE4017, GSE9348, GSE4183, GSE8671, GSE20916) for CRC, three (GSE48964, GSE29718, GSE55205) for obesity and three (GSE7429, GSE2208, GSE7158) for osteoporosis. Through meta-analysis, we obtained 9058 significant differentially expressed genes for CRC, 275 for obesity and 91 for osteoporosis. CRC and obesity shared a total of 192 genes. Among them, we found genes on insulin signaling pathways, such as PDK1, PRKAG2 and PDE3B, and adipocytokines, such as IL6 and IL8.

The three diseases osteoporosis, obesity and CRC shared six genes. Table \ref{genes} lists the genes and literature evidences, which support their relationships with each of the three diseases. Among them, FOS, JUN, and FOSB are oncogenes. FOS and JUN are known on the insulin signaling pathway. FOSB is on the AP1 pathway, which is associated with the proliferation of colon cancer cells \cite{ashida2005ap}.
Several studies suggested that overexpression of FOSB increases the responding of high fat reward while decreases energy expenditure and promotes adiposity \cite{thakali2014maternal,vialou2011role}.
\begin{sidewaystable}[h]
\caption{C\lowercase{OMMON GENES SHARED BY OBESITY, COLORECTAL CANCER AND OSTEOPOROSIS, AND PLAUSIBLE EVIDENCE SUPPORTING THEIR RELATIONSHIPS WITH THE THREE DISEASES}.}
\label{genes}
\centering
\begin{tabular}{l|m{4.5cm}|m{4.5cm}|m{4.5cm}}
GENES     & OBESITY                                                                                                                                                   & CRC                                                                                                 & OSTEOPOROSIS                                                                                                                        \\\hline
PPP1R15A* & In the bone morphogenetic protein (BMP) signaling pathway, which regulates appetite \cite{townsend2012bone}                                                              & Mutations in the BMP pathway are related with colorectal carcinogenesis \cite{hardwick2008bone}                 & In the bone morphogenetic protein signaling pathway, which are associated with bone-related diseases, such as osteoporosis \cite{chen2012tgf} \\
FOS       & diet-induced obesity is accompanied by alteration of FOS expression \cite{parker2013glucagon}                                                                    & Proto-oncogene, in the KEGG pathway of colorectal cancer \cite{kanehisa2000kegg}                                   & Mice lacking c-fos develop severe osteopetrosis \cite{okada1994mice}                                                                         \\
FOSB      & positive association between maternal obesity \cite{thakali2014maternal}                                                                                                   & Oncogene, regulators of cell proliferation, has a debatable impact on CRC patient survival \cite{pfannschmidt2009identification} & Overexpression of FosB increases bone formation \cite{sabatakos2000overexpression}                                                                           \\
HADHA*    & Associated with multiple fatty acid metabolism pathways \cite{liberzon2011molecular}                                                                                          & Unknown. Associated with breast cancer \cite{mamtani2012association}                                                    & Unknown.                                                                                                                            \\
JUN       & The c-Jun NH2-terminal Kinase Promotes Insulin Resistance \cite{aguirre2000c}                                                                                      & Proto-oncogene, in the KEGG pathway of colorectal cancer \cite{kanehisa2000kegg}                                 & Associated with osteogenesis \cite{lewinson2003stimulation,krzeszinski2014mir}                                                                                           \\
NRIP1*    & Down-regulated in obese subjects, may suggest a compensatory mechanism to favor energy expenditure and reduce fat accumulation in obesity states \cite{catalan2009rip140} & Unknown. Involved in regulation of E2F1, an oncogene \cite{docquier2010transcriptional}                                      & Modulates transcriptional activity of the estrogen receptor. Interact with ESR1 and ESR2 in osteoporosis \cite{moron2006multilocus}\\\hline
\end{tabular}
\end{sidewaystable}


Interestingly, we found several genes not involving insulin signaling. Gene PPP1R15A is in the bone morphogenetic
protein signaling (BMP) pathway and its superfamily, the TGF beta signaling pathway. The mutation of BMP pathway has been found in patients with juvenile polyposis, which is rare syndrome with an increased risk for developing CRC \cite{howe2001germline,brosens2007risk}. Mutations in TGF beta signaling also have been found susceptibility to CRC through genome-wide association studies \cite{bellam2010tgf}. A recent mouse experiment also showed that the BMP pathway regulates brown adipogenesis, energy expenditure and appetite, thus is highly associated with diet-induced obesity \cite{townsend2012bone}. These evidences support our result. Further investigation is required to confirm and elucidate the role of the BMP pathway in the connection between obesity and CRC.

Gene NRIP1 regulates the estrogen receptor. Its interaction with sex hormone receptors plays a role in both obesity \cite{catalan2009rip140} and osteoporosis \cite{moron2006multilocus}. Its relationship with CRC is unclear yet, but studies suggested that estrogen may have protective effect on CRC \cite{barzi2013molecular}. Gene HADHA is on multiple pathways of fatty acid metabolism. But its role in CRC and osteoporosis is unknown yet.

To identify the common genes among obesity, CRC and osteoporosis, we currently analyzed the gene expression data, which can be noisy. While we found literature evidences to support the detected genes and their relationships with both obesity and CRC, these candidate genes need further investigations, for example, through mouse model experiments.

\section{D\lowercase{ISCUSSION}}
The genetic connection between CRC and obesity is multifactorial and inconclusive. In this study, we developed a comorbidity network analysis approach, which suggested that  osteoporosis is important for the connection between obesity and CRC. We identified common genes among obesity, CRC and osteoporosis, and found these genes are associated with the regulation of sex hormone receptors and growth factors inducing bone formation. These genes are candidates in explaining the genetic overlaps between obesity and CRC.

Our comorbidity network may be not inclusive and biased toward the diseases whose drugs have high toxicity. The FDA adverse event reporting system collects data from medical product manufacturers, health professionals, and the public. The diseases without drug treatments are not included in the data, and the disease comorbidity relationships were often under-estimated in practice based on these data. In this study, we developed a network analysis approach to compensate the bias of the comorbidity data. In the future, including more complete patient disease data may facilitate the detection of new interesting comorbidities other than osteoporosis for obesity and CRC.

In addition, we currently detect comorbidities based on disease co-occurrence. The co-occurrence patterns may indicate the increase of the risk between two diseases in a mutual way. Incorporating more comprehensive patient-level data, such as time series data, may help refine the disease relationships and control confounding factors.

\section{Conclusions}
We constructed a disease comorbidity network through mining large scale patient data.
We developed an approach to analyze the comorbidity network and detect shared comorbidities between two diseases.
Using this approach, we identified osteoporosis as an important comorbidity for both
CRC and obesity. We discovered the common genes among obesity, CRC and osteoporosis,
and found these genes are associated with the regulation of sex hormone receptors and growth factors inducing bone formation.
We showed that these genes have the potential to explain the genetic overlaps between obesity and CRC.

