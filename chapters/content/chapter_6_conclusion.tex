\chapter{Conclusions and future work}\label{conclusion}
\section{Conclusions}
The dissertation focused on localizing faults in numerical softwares with causal inference technique.  The coverage based CSFL techniques are  poorly suited to localize faults in numerical softwares for two reasons: (1) the positivity condition may be violated in the tests (2) numerical programs often have relative few conditional branches.

For the positivity violation problem in CSFL, we proved that two types of violations of positivity: structural violations and random violations, do exist in Baah et al’s CSFL technique. We established that structural violations are not harmful, but random violations can distort suspiciousness scores.  We proposed a modification to the way suspiciousness scores are assigned with Baah et al.’s technique to address random violations.  Our empirical results show that it improves the performance of Baah et al.’s technique. We also presented a probabilistic characterization of Baah et al’s estimator which is more efficient way to compute the same scores. 

For the fault localization in numerical softwares, we proposed two models: NUMFL and BART.  In NUMFL model, we employed two different propensity scores, GPS and CBPS, to control confounding bias involving floating-point program variables that carry erroneous values to correct statements.  Then a quadratic regression model is used to estimate the AFCE of a numerical expression. The empirical results show that NUMFL is more effective than five well known SFL baseline techniques. Also GPS is more effective in controlling confounding bias comparing to CBPS. In BART model, we applied Bayesian additive regression trees to approximate the dose response function (DRF) of both the treatment variable and the confounding variables. Then we proposed an average causal effect estimation method based on BART for continuous treatment variables, which does not require techniques like propensity scores to control confounding bias. The empirical results show that BART is superior than NUMFL and other five baselines in fault localization of numerical software with single faults. In experiments, we found that the performance of BART is a robust to the training data size and the number of trees. We also discussed the computation time of NUMFL and BART. 

We extended our research to a special numerical software, embedded control systems, whose observational data are time series. We use dynamic Bayesian networks (DBNs) to model the time-evolution of the state space of the system. For fault localization, we improved our prior work and proposed FLECS 2.0. FLECS 2.0 use a DBN to detect the anomalous variables from the data of failure trajectories, and then assigned a suspiciousness score to each variable of the control code of the embedded system. Our experiment use two medical robot prototypes developed in our lab. We developed a high-level software system to monitor the state of the robots and collect data for experiments. The empirical results show that the our technique is more effective than two baselines in localizing faults in embedded control systems.

\section{Future work}
In the future work, will seek to extend our empirical results to a broader range of subject programs with more varied fault types.  For NUMFL, we intend eventually to evaluate NUMFL in a user study, but given the difficulty of conducting an unbiased one, we think it is desirable to refine NUMFL as much as possible beforehand.  Finally, we will also explore the integration of NUMFL with coverage or predicate based causal SFL techniques. For BART model, we will seek to extend BART model to localize faults in non numerical softwares with statement coverage or predicate information. For FLECS 2.0, we plan to collect more empirical data on the performance of our technique, as well as explore causal inference techniques to improve the effectiveness of the technique. Also, if more embedded systems are available, we could test our technique on them. 










