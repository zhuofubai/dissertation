\chapter{Conclusions and future work}\label{conclusion}
\section{Conclusions}
As the biomedical data become big, complex, and heterogeneous, 
we need computational approaches to combine different kinds of data
and discover new knowledge from them. A major challenge to developing
computational approaches for biomedical applications is to ask the right
question, gather relevant data and design algorithms based on the understanding
of specific problems. In this dissertation, I present
a knowledge guided strategy towards addressing this challenge.
I use problem-specific domain knowledge to guide the data gathering,
data fusion and algorithm design. I demonstrate the effectiveness of the strategy 
using the applications of disease image retrieval (Chapter \ref{image2}), 
disease gene prediction (Chapter \ref{chap:malaria}, \ref{phenotype}, and \ref{cancer}), 
and drug discovery (Chapter \ref{drug}).

Chapter \ref{image2} presents a disease image
retrieval method based on organ detection towards building 
a patient-oriented health image database. We used the knowledge 
of the affected body parts for each disease to guide the disease 
image retrieval. Compared with standard supervised classification,
which trains a classifier for each disease, our approach significantly
reduces manual labeling efforts by reusing a set of pre-trained
organ detectors across multiple diseases. In addition, our method 
improves the image retrieval precision for complex diseases that
affect multiple body parts. The resulting health image database 
is automatically annotated using terms from standard medical 
ontologies and will create a rich source of information to support
patient education and decision making.

Chapter \ref{chap:malaria}, \ref{phenotype}, and \ref{cancer} introduce
disease gene prediction approaches for parasitic infectious diseases,
multifactorial diseases, and cancers. We used domain knowledge to
guide the construction of disease specific gene prediction models using unique data. 
In Chapter \ref{chap:malaria}, 
we constructed a cross-species genetic network to model the interaction
between human and pathogen, and prioritized 
disease associated genes using network analysis.
We applied the approach on {\it Plasmodium falciparum} malaria. 
Our approach predicted both known and novel genes that are 
associated with malaria pathogenesis, and the predicted genes have
translation potential in anti-malaria drug discovery. 

In Chapter \ref{phenotype}, we constructed a new disease phenotype network
from a unique data source of human disease phenotype. We then 
designed an innovative strategy to predict disease associated genes
from combined multiple different phenotype networks and genetic networks.
Our approach achieved significantly improved the performance
comparing with the gene prediction approach using only one phenotype
data source. We applied the approach on Crohn's disease and demonstrated that the gene prediction result
has translational potentials to guide drug discovery.

In Chapter \ref{cancer}, we constructed a disease comorbidity network 
through mining large scale patient data. We developed an approach to 
analyze the comorbidity network and detect shared comorbidities between two diseases.
Using this approach, we identified osteoporosis as an important comorbidity for both
colorectal cancer and obesity. We discovered the common genes among the three diseases,
and showed that these genes have the potential to explain the genetic overlaps between obesity and colorectal cancer.

Finally, Chapter \ref{drug} presents a novel drug repositioning approach combining 
both disease genetics knowledge and mouse model phenotypes. We applied the approach
to predict new drugs for Parkinson's disease (PD). Our approach can identify FDA-approved 
PD drugs and prioritize novel PD drugs. Our approach outperformed a recently proposed 
drug discovery method using the mouse phenotype data. Gene expression analysis 
on the top-ranked candidate drugs suggested quetiapine as a potential PD therapy.

\section{Future work}
\subsection{Disease image retrieval}
For the future work, we plan to train more organ detectors and apply the
method to handle more diseases. To cover a wider range of diseases,
we plan to use texture pattern recognition to further improve the
retrieving precision in detecting organs that do not appear as concrete objects in images, 
such as skin, muscle, and veins.
For disease terms that have no body-site information in the ontologies, 
we plan to extend our approach by scanning the web images with all organ detectors. 

\subsection{Disease gene prediction}
In this dissertation, I demonstrated that the genes predicted by the proposed 
computational approaches have the potential in drug discovery. 
One nature subsequent work is to develop drug repositioning methods
through matching the targets of approved drugs to predicted genes.
We plan to analyze the functions of the top-ranked predicted genes
and further filter the strong candidate drug target genes.
We will also systematically integrate the gene function and drug action data
into the drug discovery approach.



\subsection{Drug repositioning}
Currently, we combined the disease genetics and mouse phenotype
data to predict new drug indications. Specifically, we used the disease genes
in OMIM as the seeds to detect PD mouse phenotypes.
In the future, we will incorporate more data, such as human disease phenotypes, 
disease similarities and drug similarities in the drug repositioning approach to 
further prioritize strong candidate drugs.  We will also 
develop approaches to combine other disease genetics data, including disease 
associated genes in GWAS and prediction results from computational approaches 
in our approach.
The future work also includes developing methods to validate the
predicted drugs using additional data, such as gene expression level changes associated with
the diseases and the drug compounds.










